\documentclass[conference]{IEEEtran}
\IEEEoverridecommandlockouts
% The preceding line is only needed to identify funding in the first footnote. If that is unneeded, please comment it out.
\usepackage{cite}
\usepackage{amsmath,amssymb,amsfonts}
\usepackage{algorithmic}
\usepackage{graphicx}
\usepackage{textcomp}
\usepackage{xcolor}
\usepackage{indentfirst}
\def\BibTeX{{\rm B\kern-.05em{\sc i\kern-.025em b}\kern-.08em
    T\kern-.1667em\lower.7ex\hbox{E}\kern-.125emX}}
\begin{document}

\title{Aplicação da Lógica Fuzzy na Previsão do Campeonato Brasileiro de Futebol de 2024 \\}

\author{
    \IEEEauthorblockN{
        1\textsuperscript{º} Marcelo Andrade de Jesus Filho
    }
    \IEEEauthorblockA{
        \textit{Engenharia de Sistemas} \\
        \textit{Universidade Federal de Minas Gerais}\\
        Belo Horizonte, Brasil \\
        marcelo.anje@outlook.com
    }
    \and
    \IEEEauthorblockN{
        2\textsuperscript{º} Vinicius Cardoso Antunes
    }
    \IEEEauthorblockA{
        \textit{Engenharia de Sistemas} \\
        \textit{Universidade Federal de Minas Gerais}\\
        Belo Horizonte, Brasil \\
        vinicius.mct17@gmail.com
    }
}

\maketitle

\begin{abstract}
Esse artigo apresenta uma abordagem matemática para a previsão dos resultados do 
Campeonato Brasileiro de Futebol de 2024 utilizando lógica fuzzy. 
A lógica fuzzy é uma técnica que lida com a incerteza e imprevisão, aplicando 
modelos e prevendo resultados dos confrontos entre times a partir de diferentes 
variáveis. O estudo envolve a coleta e a análise de dados de partidas dos anos 
anteriores disponíveis na internet, fornecendo dados sobre condições dos times, 
atributos dos jogadores e informações sobre cada partida.
Através da definição de funções de pertinência e regras, é criado um sistema de 
inferência fuzzy, sendo possível gerar um resultado de cada partida, e por fim, 
o resultado do campeonato. 
Os resultados obtidos demonstram a eficácia da lógica fuzzy na previsão de 
cenários, sendo uma ferramenta útil para usos científicos e comerciais. 
\end{abstract}

\begin{IEEEkeywords}
Campeonato, Futebol, Fuzzificação, Modelagem, Previsão, Sistema, Esporte, Lógica
\end{IEEEkeywords}

\section{Introdução}
\indent Desde os primórdios da civilização, os seres humanos possuem um desejo 
intrínseco de competir, característica impulsionada por uma combinação de
diversos fatores. Biologicamente, a competição pode ser atrelada à busca por 
sobrevivência e à reprodução, pois os indivíduos competiam por recursos 
escassos e por parceiros sexuais. Psicologicamente, competir é uma maneira de 
obter reconhecimentos dos outros indivíduos do grupo e ser aceito socialmente, 
além de validar habilidades. Por fim, a competição pode ser considerada uma 
característica que molda o comportamentos dos indivíduos e influencia as 
estruturas sociais e econômicas da nossa socidade \\
\indent Com o passar dos anos, as apostas esportivas crescem cada vez mais no 
Brasil, com casas de apostas presentes como patrocinadoras de praticamente 
todos as equipes da série A do Campeonato Brasileiro de 2024. Porém, as apostas 
não são um fenômeno recente, existindo registros que remontam à antiguidade, 
quando indivíduos de civilizações greco-romanas apostavam em resultados dos 
Jogos Olímpicos e corridas de bigas. Ao longo dos séculos houveram mudanças na 
maneira que as apostas eram feitas, sendo incorporadas em diferentes esportes e 
eventos. No século XIX, o Reino Unido ficou conhecido por estabelecer uma base 
de como as apostas são feitas nos dias atuais, onde haviam casas que geravam os 
valores de retorno para os diferentes tipos de apostas disponíveis. Atualmente, 
o crescimento do uso da internet possibilitou maior facilidade para que usuários 
pudessem realizar os seus palpites sem precisar sair de casa.\\
\indent Com o grande crescimento do poder computacional e avanços tecnológicos, 
foram desenvolvidos diversos algoritmos que tentam prever os resultados de 
partidas e campeonatos. Ferramente de machine learning e análise de grandes 
bases de dados, ou big data, permitem criar modelos com alta precisão. Uma 
destas diversas ferramentas é a lógica fuzzy, particularmente eficaz em modelar 
casos complexos e subjetivos, diferentes da lógica binária. Este artigo explora 
o uso dessa ferramenta na tentativa de prever os resultados do Campeonato 
Brasileiro de 2024, que com o tempo, pode provar ou não a eficácia do modelo.

\section{Revisão da Literatura}
\subsection{Sistemas Difusos}
\indent Os sistemas difusos, também conhecidos como sistemas nebulosos ou 
sistemas fuzzy, são sistemas de controle que possuem como objetivo criar modelos 
matemáticos baseados em informações imprecisas ou subjetivas, intrínsecas aos 
seres humanos. O conceito de fuzzy surgiu a partir da Teoria de Conjuntos Fuzzy, 
desenvolvida em 1965 pelo matemático estadunidense Lotfi Asker Zadeh.

\subsection{Lógica Fuzzy}
\indent Alternativamente à lógica Booleana, em que os valores de variáveis são 
somente 0 ou 1, em um sistema difuso as variáveis podem ter valores 
intermediários entre 0 e 1, que representam graus de pertinência. Na lógica 
Fuzzy, as proposições podem ser parcialmente verdadeiras ou parcialmente falsas, 
refletindo a natureza ambígua dos fenômenos do mundo real. Atualmente, a lógica 
Fuzzy possui diversas aplicações, sendo utilizada em sistemas de controle, 
sistemas de predição, sistemas de reconhecimento de padrões e sistemas de 
tomadas de decisão.

\section{Objetivos}
\indent Esse artigo tem como principais objetivos desenvolver um modelo de 
previsão baseado em lógica fuzzy para prever os resultados do Campeonato 
Brasileiro de 2024 utilizando dados disponibilizados na internet, analisar a 
eficácia desse modelo comparando com os últimos campeonatos, identificar as 
variáveis mais interessantes de serem utilizadas, validar o modelo através de 
testes, alterando variáveis, comparando os resultados e explorando diferentes 
cenários.\\
\indent A modelagem desse problema pode ser dividida em quatro partes, sendo o 
tratamento dos dados utilizados através da busca por variáveis mais 
relevantes e combinação de dados de diferentes anos através de ferramentas 
estatísticas, criação das funções de pertinência a partir dos dados gerados, 
criação das regras com base nas variáveis e funções de pertinência e simulação 
e computação dos resultados.
\indent Por fim, com a finalização do projeto, é possível que a forma de modelar 
o sistema de previsão de resultados possa ser expandido para diversas outras 
áreas, como a medicina, engenharia, biologia ou mesmo em outros esportes.

\section{Metodologia}
\indent Este trabalho tem como objetivo criar um sistem de previsões para um 
campeonato. Para isso, será necessário criar uma maneira de analisar os dados 
disponíveis e criar a estrutura do sistema fuzzy necessária para que os 
resultados sejam obtidos de maneira correta.\\

\subsection{Fonte de Dados}
\indent Como fonte de dados, foi utilizado um banco de dados disponibilizado na 
internet com o nome 'Base dos Dados' em um arquivo de formato CSV 
(Comma-separated Values), que contém as seguintes colunas:
\begin{itemize}
    \item Ano
    \item Data
    \item Rodada
    \item Estádio
    \item Árbitro
    \item Público
    \item Público Máximo
    \item Time Mandante
    \item Time Visitante
    \item Técnico Mandante
    \item Técnico Visitante
    \item Colocação Mandante
    \item Colocação Visitante
    \item Valor Equipe Titular Mandante
    \item Valor Equipe Titular Visitante
    \item Idade Média Titular Mandante
    \item Idade Média Titular Visitante
    \item Gols Mandante
    \item gols Visitante
    \item Gols 1º Tempo Mandante
    \item Gols 1º Tempo Visitante
    \item Escanteios Mandante
    \item Escanteios Visitante
    \item Faltas Mandante
    \item Faltas Visitante
    \item Chutes Bola Parada Mandante
    \item Chutes Bola Parada Visitante
    \item Defesas Mandante
    \item Defesas Visitante
    \item Impedimentos Mandante
    \item Impedimentos Visitante
    \item Chutes Mandante
    \item Chutes Visitante
    \item Chutes Fora Mandante
    \item Chutes Fora Visitante
\end{itemize}

\subsection{Variáveis Utilizadas}
\indent A partir desses dados, as variáveis de entrada definidas serão as seguintes:
\begin{itemize}
    \item Valor da Equipe Mandante
    \item Valor da Equipe Visitante
    \item Idade Média da Equipe Mandante
    \item Idade Média da Equipe Visitante
    \item Média de Gols a Favor da Equipe Mandante
    \item Média de Gols Contra da Equipe Visitante
    \item Média de Gols Contra da Equipe Mandante
    \item Média de Gols a Favor da Equipe Visitante
    \item Percentual de Vitórias da Equipe Mandante
    \item Percentual de Derrotas da Equipe Mandante
    \item Percentual de Vitórias da Equipe Visitante
    \item Percentual de Derrotas da Equipe Visitante
    \item Percentual de Vitórias da Equipe Mandante
    \item Percentual de Vitórias da Equipe Visitante
\end{itemize}
Como variável de saída temos apenas o número de gols de cada equipe na partida simulada.

\subsection{Funções de Pertinência}
\indent Após as definições das variáveis, é necessário criar as funções de 
pertinência, que serão responsáveis por representar a incerteza na modelagem. Os
valores escolhidos foram:

\subsubsection{Valor da Equipe Mandante}
\begin{itemize}
    \item Baixo: [0, 30, 70]
    \item Médio: [40, 100, 110]
    \item Alto: [90, 150, 220]
\end{itemize}

\subsubsection{Valor da Equipe Visitante}
\begin{itemize}
    \item Baixo: [0, 30, 70]
    \item Médio: [40, 100, 110]
    \item Alto: [90, 150, 220]
\end{itemize}

\subsubsection{Idade Média do Time Mandante e Visitante}
\begin{itemize}
    \item Jovem: [20, 20, 23]
    \item Média: [23, 25, 27]
    \item Experiente: [26, 29, 31]
\end{itemize}

\subsubsection{Média de Gols Marcados e Sofridos pelo Time Mandante}
\begin{itemize}
    \item Baixa: [0, 0.6, 1.2]
    \item Média: [1, 1.5, 1.8]
    \item Alta: [1.5, 1.85, 2.2]
\end{itemize}

\subsubsection{Média de Gols Marcados e Sofridos pelo Time Visitante}
\begin{itemize}
    \item Baixa: [0, 0.4, 1]
    \item Média: [0.85, 1, 1.1]
    \item Alta: [1, 1.2, 1.4]
\end{itemize}

\subsubsection{Percentual de Vitórias e Derrotas do Time Mandante}
\begin{itemize}
    \item Baixo: [0, 32, 46]
    \item Médio: [35, 50, 60]
    \item Alto: [52, 60, 75]
\end{itemize}

\subsubsection{Percentual de Vitórias e Derrotas do Time Visitante}
\begin{itemize}
    \item Baixo: [0, 15, 20]
    \item Médio: [18, 25, 30]
    \item Alto: [28, 34, 40]
\end{itemize}

\subsubsection{Percentual de Vitórias por Colocação do Time Mandante}
\begin{itemize}
    \item Baixo: [0, 15, 40]
    \item Médio: [38, 50, 65]
    \item Alto: [60, 85, 100]
\end{itemize}

\subsubsection{Percentual de Vitórias por Colocação do Time Visitante}
\begin{itemize}
    \item Baixo: [0, 15, 24]
    \item Médio: [17, 24, 32]
    \item Alto: [30, 69, 75]
\end{itemize}

\subsubsection{Gols Marcados no Jogo}
\begin{itemize}
    \item Poucos: [0, 0, 2]
    \item Moderados: [0, 2, 4]
    \item Muitos: [3, 4, 7]
\end{itemize}


\subsection{Regras Definidas}
\indent As regras foram definidas de modo a estimar a quantidade de gols em uma 
partida de futebol com base nas várias variáveis de entrada apresentadas, como 
o valor da equipe mandante e visitante, a idade média dos times mandante e 
visitante, a média de gols marcados e sofridos pelos times, o percentual de 
vitórias e derrotas dos times mandante e visitante, e o percentual de vitórias 
por colocação dos times mandante e visitante.

Cada regra fuzzy é definida com uma série de condicionais e consequências. 
Exemplificando, a regra 1 estabelece que se o valor da equipe mandante for alto 
e o valor da equipe visitante também for alto, a quantidade de gols será 
considerada baixa, pois são equipes similares em um aspecto. Assim, as regras 
abrangem diversas combinações possíveis entre as variáveis de entrada para 
prever a quantidade de gols esperada na partida.

\subsection{Simulação}
\indent Por fim, criadas as condições para a simulação do sistema, podemos 
realizar a execução, que trará como resultado a quantidade de gols de cada 
equipe em cada partida. Ao todo serão 380 partidas divididas em 38 rodadas, que 
terão como possíveis resultados a vitória do time mandante ou visitante, 
resultando em 3 pontos para o time vencedor e 0 pontos para o time perdedor ou 
um empate, que resultará em 1 ponto para cada equipe. No final do campeonato, a 
classificação é dada pelo número de pontos, havendo como critério de desempate o 
número de vitórias e o saldo de gols.

\section{Resultados}

\subsection{Resultados dos Campeonatos Anteriores}

\indent Antes de analisar os resultados, podemos ter como referência os 4 
primeiros colocados do Campeonato Brasileiro nos últimos 5 anos, presentes na 
tabela abaixo:

\begin{table}[htbp]
    \caption{Colocações dos 4 primeiros times nos últimos 5 campeonatos}
    \begin{center}
        \begin{tabular}{|c|c|c|c|c|}
        \hline
        \textbf{Tabela} & \multicolumn{4}{|c|}{\textbf{Ano}} \\
        \cline{1-5} 
        \textbf{Pos.} & \textbf{2023} & \textbf{2022} & \textbf{2021} & \textbf{2020}\\
        \hline
        1º & Palmeiras & Palmeiras & Atlético-MG & Flamengo\\
        \hline
        2º & Grêmio & Internacional & Flamengo & Internacional\\
        \hline
        3º & Atlético-MG & Fluminense & Palmeiras & Atlético-MG\\
        \hline
        4º & Flamengo & Corinthians & Fortaleza & São Paulo\\
        \hline
    \end{tabular}
    \label{tab1}
    \end{center}
\end{table}

\indent A partir desses dados, é possível observar que times como Palmeiras, Atlético-MG 
e Flamengo estão presentes frequentemente nas melhores colocações nos últimos 
anos, de modo a se esperar que a simulação tenda a continuar classificando estes 
clubes nas melhores colocações. 

\subsection{Resultado da Simulação}
\indent Classificação final para o Campeonato Brasileiro de 2024:

\begin{table}[htbp]
    \caption{Resultado da simulação (previsão)}
    \begin{center}
        \begin{tabular}{|c|c|c|}
        \cline{1-3}
        \textbf{Posição} & \textbf{Clube} & \textbf{Pontos}\\
        \hline
        1º & Flamengo & 110\\
        \hline
        2º & Palmeiras & 100\\
        \hline
        3º & Atlético-MG & 88\\
        \hline
        4º & Corinthians & 88\\
        \hline
        5º & Internacional & 78\\
        \hline
        6º & Athletico-PR & 73\\
        \hline
        7º & Fluminense & 68\\
        \hline
        8º & São Paulo & 67\\
        \hline
        9º & RB Bragantino & 56\\
        \hline
        10º & Cruzeiro & 55\\
        \hline
        11º & Grêmio & 52\\
        \hline
        12º & Botafogo & 34\\
        \hline
        13º & Cuiabá-MT & 32\\
        \hline
        14º & Atlético-GO & 29\\
        \hline
        15º & EC Bahia & 26\\
        \hline
        16º & Vasco da Gama & 26\\
        \hline
        17º & Fortaleza & 23\\
        \hline
        18º & Criciúma EC & 15\\
        \hline
        19º & EC Vitória & 9\\
        \hline
        20º & Juventude & 7\\
        \hline
    \end{tabular}
    \label{tab2}
    \end{center}
\end{table}

\indent Como citado anteriormente, a presença de clubes como Palmeiras, 
Atlético-MG e Flamengo era esperada pois eram clubes que estão com resultados 
consistentes nos últimos anos.

\subsection{Possíveis Problemas}
\indent Devido ao fato do modelo não ser perfeito e ao futebol brasileiro ser 
muito imprevisível, é possível que os resultados sejam muito divergentes dos 
resultados reais. O modelo não considera fatores como troca de técnicos, 
problemas financeiros das equipes, lesões de jogadores ou desastres naturais, 
tornando-o impreciso por natureza.\\
\indent Outra consideração a ser feita é a de que em todo campeonato, os 4 últimos 
clubes na classificação são rebaixaos para a divisão abaixo, tornando os dados 
dessas equipes inexistentes em alguns anos. Uma solução seria ter os dados de 
todas as partidas de todas as divisões do Campeonato Brasileiro, porém essa 
documentação é escassa devido ao fatos das divisões menores terem menos 
relevância para o público em geral.

\section{Conclusão}
\indent Este trabalho explorou a aplicação da lógica fuzzy em um sistema de 
previsão dos resultados do Campeonato Brasileiro de 2024, mostrando ser um modelo 
eficaz de previsão e provando a aplicação da aplicação da matemática para a 
modelagem de sistemas que envolvem a incerteza e subjetividade presentes nas 
ações humanas.\\
\indent A partir da análise de dados coletados, foi possível gerar variáveis, 
funções de pertinência, regras e uma simulação que resultou na previsão da 
classificação dos clubes da série A do campeonato.\\
\indent Os resultados obtidos foram os esperados, mostrando similaridade com 
os dos últimos anos e próximo do que se espera do ano em curso, contudo, é 
importante reconhecer que o modelo possui imprecisões e pode estar distante da 
realidade que pode haver no final do campeonato.\\
\indent Por fim, este estudo teve grande contribuição para o entendimento do 
funcionamento dos sistemas difusos, assim como para as aplicações desse sistema 
em problemas atuais.
\indent \textbf{Código fonte disponível em \cite{b4}}

\begin{thebibliography}{00}
\bibitem{b1} Trawiński, Krzysztof, A Fuzzy Classification System for Prediction of the Results of the
Basketball Games, July 2010.
\bibitem{b2} Vaidehi .V, Monica .S , Mohamed Sheik Safeer .S, Deepika .M, Sangeetha .S, A Prediction System Based on Fuzzy Logic, October 2008.
\bibitem{b3} Omomule T. G., Ibinuolapo A. J., Ajayi O. O., Fuzzy-Based Model for Predicting Football Match Results, February 2020.
\bibitem{b4} V. C. et al., "fuzzy-championship-predictor", 2024. código fonte disponível no GitHub. [Online]. Disponível em: https://github.com/vinicius-cardoso/fuzzy-championship-predictor
\end{thebibliography}
\end{document}
